\documentclass[12pt]{article}

% report, book
%  Русский язык

%\usepackage{bookmark}

\usepackage[T2A]{fontenc}			% кодировка
\usepackage[utf8]{inputenc}			% кодировка исходного текста
\usepackage[english,russian]{babel}	% локализация и переносы
\usepackage[title,toc,page,header]{appendix}
\usepackage{amsfonts}
\usepackage{hyperref,bookmark}


% Математика
\usepackage{amsmath,amsfonts,amssymb,amsthm,mathtools} 
%%% Дополнительная работа с математикой
%\usepackage{amsmath,amsfonts,amssymb,amsthm,mathtools} % AMS
%\usepackage{icomma} % "Умная" запятая: $0,2$ --- число, $0, 2$ --- перечисление

\usepackage{cancel}%зачёркивание
\usepackage{braket}
%% Шрифты
\usepackage{euscript}	 % Шрифт Евклид
\usepackage{mathrsfs} % Красивый матшрифт


\usepackage[left=2cm,right=2cm,top=1cm,bottom=2cm,bindingoffset=0cm]{geometry}
\usepackage{wasysym}

%размеры
\renewcommand{\appendixtocname}{Приложения}
\renewcommand{\appendixpagename}{Приложения}
\renewcommand{\appendixname}{Приложение}
\makeatletter
\let\oriAlph\Alph
\let\orialph\alph
\renewcommand{\@resets@pp}{\par
  \@ppsavesec
  \stepcounter{@pps}
  \setcounter{subsection}{0}%
  \if@chapter@pp
    \setcounter{chapter}{0}%
    \renewcommand\@chapapp{\appendixname}%
    \renewcommand\thechapter{\@Alph\c@chapter}%
  \else
    \setcounter{subsubsection}{0}%
    \renewcommand\thesubsection{\@Alph\c@subsection}%
  \fi
  \if@pphyper
    \if@chapter@pp
      \renewcommand{\theHchapter}{\theH@pps.\oriAlph{chapter}}%
    \else
      \renewcommand{\theHsubsection}{\theH@pps.\oriAlph{subsection}}%
    \fi
    \def\Hy@chapapp{appendix}%
  \fi
  \restoreapp
}
\makeatother
\newtheorem{theorem}{Теорема}
\newtheorem{definition}{Определение}
\newtheorem{postulat}{Постулат}
\newtheorem{utv}{Утверждение}

\title{Решение заданий\\ ОП "Квантовая теория поля, теория струн и математическая физика"\\[2cm]
Избранные главы теоретической и математической физики (2 семестр)}
\author{Коцевич Андрей Витальевич, группа Б02-920}
\date{\today\; Версия 5.0}

\begin{document}

\maketitle
\newpage
\newpage
\tableofcontents{}
\newpage
\maketitle

\section{Фазовые переходы и модель Изинга (А. Александров)}
\subsection{Задача 1}
\begin{enumerate}
    \item В теории среднего поля для каждого спина можно записать:
    \begin{equation}
        s_i=\langle s_i\rangle+\delta s_i
    \end{equation}
    где $m=\langle s_i\rangle$ -- это средняя намагниченность, а $\delta s_i$ мало.\\
    Гамильтониан (полная энергия) в модели Изинга с точностью до константы:
    \begin{equation}
        \mathcal{H}=-Jm\sum\limits_{<ij>}s_is_j-h\sum\limits_{i=1}^Ns_i
    \end{equation}
    где $J$ -- константа спин-спинового взаимодействия, $h$ -- внешнее магнитное поле. Угловые скобки означают суммирование только по ближайшим соседям.
    Слагаемые в сумме со спиновым взаимодействием $s_is_j$ равны:
    \begin{equation}
        s_is_j=(\langle s_i\rangle+\delta s_i)(\langle s_j\rangle+\delta s_j)=\langle s_i\rangle\langle s_j\rangle+\langle s_i\rangle\delta s_j+\langle s_j\rangle\delta s_i+\delta s_i\delta s_j
    \end{equation}
    Т.к. $\delta s_i$ мало, то последним членом можно пренебречь:
    \begin{equation*}
        s_is_j=m^2+m\delta s_i+m\delta s_j=m^2+m(s_i-\langle s_i\rangle)+m(s_j-\langle s_j\rangle)=m^2+m(s_i+s_j)-2m^2=m(s_i+s_j-m)
    \end{equation*}
    Следовательно, гамильтониан в теории среднего поля равен:
    \begin{equation}
        \mathcal{H}=-Jm\sum\limits_{<ij>}(s_i+s_j-m)-h\sum\limits_{i=1}^Ns_i
    \end{equation}
    Из-за симметрии $i$ и $j$ в сумме по ближайшим соседям можно записать:
    \begin{equation}
        \mathcal{H}=-Jm\sum\limits_{<ij>}(2s_i-m)-h\sum\limits_{i=1}^Ns_i
    \end{equation}
    Первая сумма достаточно сложна, попробуем упростить её. Для этого используем то, что у каждого спина одинаковое число соседей $q$ (например, в одномерной модели Изинга $q=2$, в двумерной с квадратной решёткой -- $q=4$).
    \begin{equation}
        \mathcal{H}=-\frac{qJm}{2}\sum\limits_{i=1}^{N}(2s_i-m)-h\sum\limits_{i=1}^Ns_i
    \end{equation}
    Коэффициент $\frac{1}{2}$ стоит, потому что сумма учитывает каждое парное взаимодействие дважды.
    \begin{equation}
        \mathcal{H}=\frac{qJm^2N}{2}-qJm\sum\limits_{i=1}^{N}s_i-h\sum\limits_{i=1}^Ns_i=\frac{qJm^2N}{2}-(qJm+h)\sum\limits_{i=1}^{N}s_i
    \end{equation}
    Введём эффективное внешнее магнитное поле $\boxed{h_\text{eff}=qJm+h}$ и получим конечный ответ:
    \begin{equation}            
        \boxed{\mathcal{H}=\frac{1}{2}qJm^2N-h_\text{eff}\sum\limits_{i=1}^{N}s_i}
    \end{equation}
    В двумерной модели Изинга с квадратной решёткой: $\mathcal{H}=2Jm^2N-h_\text{eff}\sum\limits_{i=1}^{N}s_i$, где $h_\text{eff}=2Jm+h$.
    \item Статистическую сумму можно найти по формуле:
    \begin{equation}
        \mathcal{Z}=\sum\limits_{MS}\exp(-\beta\mathcal{H})
    \end{equation}
    где $\beta=\frac{1}{k_B T}$, а суммирование проводится по всем возможным микросостояниям, которые обозначены MS.
    \begin{multline*}
        \mathcal{Z}=\exp\left(-\frac{1}{2}\beta qJm^2N\right)\sum\limits_{s_1=\pm 1}\sum\limits_{s_2=\pm 1}...\sum\limits_{s_N=\pm 1}\exp\left(\beta h_\text{eff}\sum\limits_{i=1}^{N}s_i\right)=\\
        =\exp\left(-\frac{1}{2}\beta qJm^2N\right)\sum\limits_{s_1=\pm 1}\exp\left(\beta h_\text{eff}s_1\right)\sum\limits_{s_2=\pm 1}\exp\left(\beta h_\text{eff}s_2\right)...\sum\limits_{s_N=\pm 1}\exp\left(\beta h_\text{eff}s_N\right)=\\
        =\exp\left(-\frac{1}{2}\beta qJm^2N\right)\prod_{i=1}^N(\exp\left(\beta h_\text{eff}\right)+\exp\left(-\beta h_\text{eff}\right))=\exp\left(-\frac{1}{2}\beta qJm^2N\right)(2\cosh{\beta h_\text{eff}})^N
    \end{multline*}
    \begin{equation}\label{eq1}
        \boxed{\mathcal{Z}=(2\cosh{\beta h_\text{eff}})^N\exp\left(-\frac{1}{2}\beta qJm^2N\right)}
    \end{equation}
    Воcпользуемся определением средней намагниченности:
    \begin{equation}
        m=\frac{1}{N}\sum\limits_{i=1}^N\langle s_i\rangle=\frac{1}{N}\frac{\sum\limits_{i=1}^N s_i\exp(-\beta\mathcal{H})}{\mathcal{Z}}
    \end{equation}
    Сумму не придётся считать, если понять, что она уже почти посчитана:
    \begin{equation}
        \sum\limits_{i=1}^N s_i\exp(-\beta\mathcal{H})=-\left(\frac{\partial\mathcal{H}}{\partial h_\text{eff}}\right)\exp(-\beta\mathcal{H})=\frac{1}{\beta}\frac{\partial(\exp(-\beta\mathcal{H}))}{\partial h_\text{eff}}=\frac{1}{\beta}\frac{\partial\mathcal{Z}}{\partial h_\text{eff}}
    \end{equation}
    \begin{equation}
        m=\frac{1}{N\beta\mathcal{Z}}\frac{\partial{\mathcal{Z}}}{\partial{h_\text{eff}}}=\frac{1}{N\beta}\frac{\partial(\ln\mathcal{Z})}{\partial{h_\text{eff}}}
    \end{equation}
    Частную производную найдём, используя (\ref{eq1}):
    \begin{equation}
        \frac{\partial(\ln\mathcal{Z})}{\partial{h_\text{eff}}}=\frac{1}{\partial{h_\text{eff}}}\left(-\frac{1}{2}\beta qJm^2N+N\ln 2+N\ln(\cosh(\beta h_\text{eff}))\right)=N\beta\tanh(\beta h_\text{eff})
    \end{equation}
    \begin{equation}\label{eq4}
        m=\tanh(\beta h_\text{eff})=\tanh(\beta(h+qJm))
    \end{equation}
    Получилось трансцендентное уравнение. Его решения зависят от $h$ и $\beta$. Рассмотрим 2 случая:\\
    \textbf{Случай 1. $h=0$ (внешнего поля нет).}\\
    \begin{equation}
        m=\tanh(\beta qJm)
    \end{equation}
    Разложим гиперболический тангенс по формуле Тейлора:
    \begin{equation}
        \tanh(\beta qJm)=\beta qJm-\frac{(\beta qJm)^3}{3}+\mathcal{O}((\beta qJm)^5)
    \end{equation}
    Пренебрегая $\mathcal{O}((\beta qJm)^5)$, получим уравнение
    \begin{equation}
        m=\beta qJm-\frac{(\beta qJm)^3}{3}
    \end{equation}
    \begin{equation}
        m\left(1-\beta qJ-\frac{(\beta qJ)^3}{3}m^2\right)=0
    \end{equation}
    \begin{equation}\label{eq3}
        m_1=0, \quad m_{2,3}=\pm\sqrt{3\frac{1-\beta qJ}{(\beta qJ)^3}}
    \end{equation}
    2 и 3 корни существуют, если $1-\beta qJ\geq 0$. Т.е. при $k_B T\geq qJ$ решение одно, а при $k_B T\leq qJ$ решения три. Значит, критическая температура:
    \begin{equation}\label{eq2}
        \boxed{T_c=\frac{qJ}{k_B}}
    \end{equation}
    В одномерном случае модели Изинга (\ref{eq2}) предсказывает, что $T_c=\frac{2J}{k_B}$. Это неверно, т.к. при конечных температурах фазового перехода нет.\\
    \textbf{Случай 2. $h\neq 0$ (внешнее поле есть).}\\
    \begin{equation}
        m=\tanh(\beta(h+qJm))
    \end{equation}
    Разложим гиперболический тангенс в ряд:
    \begin{equation}
        \tanh(\beta (h+qJm))=\beta (h+qJm)+\mathcal{O}((\beta (h+qJm))^3)
    \end{equation}
    Пренебрегая $\mathcal{O}((\beta (h+qJm))^3)$, получим уравнение
    \begin{equation}
        m=\beta (h+qJm)\rightarrow m=\frac{\beta h}{1-\beta qJ}
    \end{equation}
    При любых конечных температурах решение одно. В этом случае фазовый переход отсутствует и критической температуры нет.\\
    Найдём критические показатели $\beta$ и $\gamma$. Их определения:
    \begin{equation}
        m(t)\sim(-t)^\beta,\quad \chi(t)\sim|t|^\gamma
    \end{equation}
    где $t=\frac{T-T_c}{T_c}$.\\
    По формуле (\ref{eq3}) при $T\leq T_c$: $m=0$, при $T\geq T_c$:
    \begin{equation}
        m(T)=\left(3\frac{T^2}{T_c^3}(T_c-T)\right)^\frac{1}{2}\rightarrow \boxed{\beta=\frac{1}{2}}
    \end{equation}
    Найдём магнитную восприимчивость. По определению:
    \begin{equation}
        \chi(T)=\left(\frac{\partial m}{\partial h}\right)_{T,h=0}
    \end{equation}
    По формуле (\ref{eq4}):
    \begin{equation}
        \chi=\frac{\beta(1+qJ\chi)}{\cosh^2(\beta(h+qJm))}
    \end{equation}
    Выразим $\chi$:
    \begin{equation}
        \chi=\frac{\beta}{\cosh^2(\beta(h+qJm))-\beta qJ}
    \end{equation}
    Подставим $h=0$ и разложим гиперболический косинус в ряд:
    \begin{equation}
        \cosh(\beta qJm)=1+\mathcal{O}((\beta qJm)^2)
    \end{equation}
    Пренебрежём $\mathcal{O}((\beta qJm)^2$ при $T\rightarrow T_c$ и $T\leq T_c$ (т.к. $m\rightarrow 0$):
    \begin{equation}
        \chi=\frac{\beta}{1-\beta qJ}=\frac{1}{k_BT_c|t|}\rightarrow \boxed{\gamma=-1}
    \end{equation}
    Как видно, показательные коэффициенты в модели среднего поля отличаются от соответствующих коэффициентов в двумерной модели Изинга: $\beta=\frac{1}{8}$ и $\gamma=-\frac{7}{4}$.
    \item Уравнение состояния газа Ван-дер-Ваальса (при $\nu=1$):
    \begin{equation}\label{eq5}
        p=\frac{k_BT}{v-b}-\frac{a}{v^2}
    \end{equation}
    где $v=\frac{V}{N}$ -- объём, занимаемый одной частицей.
    \begin{equation}
        pv^3-(pb+k_BT)v^2+av-ab=0
    \end{equation}
    В критической точке все три корня уравнения сливаются в один, поэтому предыдущее уравнение эквивалентно следующему:
    \begin{equation}
        (v-v_c)^3=0
    \end{equation}
    Приравняв коэффициенты при соответствующих степенях $v$, получим значения критических параметров:
    \begin{equation}
        v_c=3b,\quad p_c=\frac{a}{27b^2},\quad T_c=\frac{8a}{27bk_B}
    \end{equation}
    Приведённые параметры определяются как отношения:
    \begin{equation}
        \varphi=\frac{v}{v_c},\quad \pi=\frac{p}{p_c},\quad \tau=\frac{T}{T_c}
    \end{equation}
    C их помощью можно переписать (\ref{eq5}) и получить приведённое уравнение Ван-дер-Ваальса:
    \begin{equation}
        \pi=\frac{8\tau}{3\varphi-1}-\frac{3}{\varphi^2}
    \end{equation}
    При $\tau<1$ существуют 2 стабильных решения (жидкость и газ):
    \begin{equation}
        \pi=\frac{8\tau}{3\varphi_\text{gas}-1}-\frac{3}{\varphi^2_\text{gas}}=\frac{8\tau}{3\varphi_\text{liquid}-1}-\frac{3}{\varphi^2_\text{liquid}}
    \end{equation}
    Решим уравнение относительно $\tau$:
    \begin{equation}
        \tau=\frac{(3\varphi_\text{gas}-1)(3\varphi_\text{liquid}-1)(\varphi_\text{gas}+\varphi_\text{liquid})}{8\varphi^2_\text{gas}\varphi^2_\text{liquid}}
    \end{equation}
    Будем приближаться к критической точке: $\varphi_\text{gas}\rightarrow 1$ и $\varphi_\text{liquid}\rightarrow 1$. Обозначим $\Delta\varphi=\varphi_\text{gas}-\varphi_\text{liquid}$, будем подходить к критической точке симметрично: $\varphi_\text{gas}=1+\frac{\Delta\varphi}{2}$ и $\varphi_\text{liquid}=1-\frac{\Delta\varphi}{2}$.
    \begin{multline*}
        \tau=\frac{(2+\frac{3\Delta\varphi}{2})(2-\frac{3\Delta\varphi}{2})}{4(1+\frac{\Delta\varphi}{2})^2(1-\frac{\Delta\varphi}{2})^2}=\left(1-\frac{9(\Delta\varphi)^2}{16}\right)\left(1-\frac{(\Delta\varphi)^2}{4}\right)^{-2}+\mathcal{O}((\Delta\varphi)^4)=\\
        =1-\frac{1}{16}(\Delta\varphi)^2+\mathcal{O}((\Delta\varphi)^4)
    \end{multline*}
    \begin{equation}
        \boxed{v_\text{gas}-v_\text{liquid}\sim(T_c-T)^\beta,\quad \beta=\frac{1}{2}}
    \end{equation}
    Определение сжимаемости:
    \begin{equation}
        \kappa=-\frac{1}{v}\left(\frac{\partial v}{\partial p}\right)_T
    \end{equation}
    Из уравнения состояния газа Ван-дер-Ваальса (\ref{eq5}):
    \begin{equation}
        \frac{\partial p}{\partial v}(v_c)=-\frac{k_BT}{(v_c-b)^2}+2\frac{a}{v_c^3}=-\frac{k_BT}{4b^2}+\frac{2a}{27b^3}=-\frac{k_BT}{4b^2}+\frac{k_BT_c}{4b^2}=-\frac{k_B}{4b^2}(T-T_c)
    \end{equation}
    \begin{equation}
        \boxed{\kappa\sim(T-T_c)^{\gamma},\quad \gamma=-1}
    \end{equation}
    Как видно, показательные коэффициенты совпадают с теорией среднего поля. Это удивительно, так как модели частиц и взаимодействий между ними совсем разные. Видимо, при фазовом переходе всё это становится несущественным. В этом состоит гипотеза универсальности.
\end{enumerate}
\subsection{Задача 2}
Запишем статистические суммы для высокой температуры для треугольной и гексагональной решёток по аналогии со случаем квадратной решётки:
\begin{equation}
    \mathcal{Z}_\triangle=2^{N_\triangle}(\cosh K)^{q_\triangle}\sum\limits_LG_\triangle(L,N_\triangle)\tanh^L K
\end{equation}
\begin{equation}
    \mathcal{Z}_{\hexagon}=2^{N_{\hexagon}}(\cosh K)^{q_{\hexagon}}\sum\limits_LG_{\hexagon}(L,N_{\hexagon})\tanh^L K
\end{equation}
где $K=\beta J$, $N_{\triangle,\hexagon}$ -- количество узлов в решётке, $q_{\triangle,\hexagon}$ -- количество ближайших соседей, $G_{\triangle,\hexagon}(L,N_{\triangle,\hexagon})$ -- количество способов нарисовать замкнутый граф длиной $L$ в решётке с $N_{\triangle,\hexagon}$ узлами.\\
Для низкой температуры:
\begin{equation}
    \mathcal{Z}_\triangle=2\exp(N_\triangle K)\sum\limits_LM_\triangle(L,N_\triangle)\exp(-2KL)
\end{equation}
\begin{equation}
    \mathcal{Z}_{\hexagon}=2\exp(N_{\hexagon} K)\sum\limits_LM_{\hexagon}(L,N_{\hexagon})\exp(-2KL)
\end{equation}
где $M_{\triangle,\hexagon}(L,N_{\triangle,\hexagon})$ -- количество доменных стенок длины $L$ в решётке с $N_{\triangle,\hexagon}$ узлами. Выразим их через $G_{\triangle,\hexagon}(L,N_{\triangle,\hexagon})$.\\
Воспользуемся двойственностью между треугольной и гексагональной решётками. Замкнутые графы соответствуют доменным стенкам на дуальной решётке. Чтобы получить гексагональную решетку из треугольной, нужно поместить узел решетки в центр каждого треугольника. Следовательно, двойственная треугольная решетка с $N_\triangle$ узлами является гексагональной решеткой с $2N_\triangle$ узлами:
\begin{equation}
    M_\triangle(L,N_\triangle)=G_{\hexagon}(L,2N_\triangle)
\end{equation}
\begin{equation}
    M_{\hexagon}(L,N_{\hexagon})=G_\triangle\left(L,\frac{N_{\hexagon}}{2}\right)
\end{equation}
Для дуальных решёток выполняется соотношение ("звезда-треугольник"):
\begin{equation}
    C\exp(K'(s_1s_2+s_2s_3+s_3s_1))=2\cosh(K(s_1+s_2+s_3))
\end{equation}
где $K$ и $K'$ -- константы для оригинальной и дуальной решёток.\\
Пусть $s_1=s_2=s_3=1$, тогда $C\exp(3K')=2\cosh(3K)$; пусть $s_1=s_2=1$, $s_3=-1$, тогда $ C\exp(-K')=2\cosh K$. Из 2 соотношений получим:
\begin{equation}
    K'=\frac{1}{4}\log\left(\frac{\cosh 3K}{\cosh K}\right),\quad C=2\sqrt[4]{(\cosh K)^3\cosh 3K}
\end{equation}
Преобразование <<звезда-треугольник>> связывает модель на гексагональной решетке с $N_{\hexagon}$ узлами и параметром $K$ треугольной моделью с $\frac{1}{2}N_{\hexagon}$ узлами и параметром $K'$:
\begin{equation}
    K\sim K'=\frac{1}{4}\log\left(\frac{\cosh 3K}{\cosh K}\right)
\end{equation}
Двойственность Крамерса-Вранье:
\begin{equation}
    K\sim K''=-\frac{1}{2}\log(\tanh K)
\end{equation}
Объединяя преобразования, получим
\begin{equation}
    K=\frac{1}{4}\log\left(\frac{\cosh(-\frac{3}{2}\log(\tanh K))}{\cosh(-\frac{1}{2}\log(\tanh K))}\right)
\end{equation}
Решая данное трансцендетное уравнение, получим
\begin{equation}
    K_c^\triangle=\frac{\log 3}{4}\rightarrow \boxed{T_c^\triangle=\frac{4J}{k_B\log 3}}
\end{equation}
Записав преобразования в обратном порядке, получим
\begin{equation}
    K=-\frac{1}{2}\log(\tanh\left(\frac{1}{4}\log\left(\frac{\cosh 3K}{\cosh K}\right)\right)
\end{equation}
Решая данное трансцендетное уравнение, получим
\begin{equation}
     K_c^{\hexagon}=\frac{1}{2}\log(2+\sqrt{3})\rightarrow \boxed{T_c^{\hexagon}=\frac{2J}{k_B\log(2+\sqrt{3})}}
\end{equation}
\section{XXX-цепочка и анзац Бете (Д. Менской)}
\subsection{Тензорное произведение}
\begin{enumerate}
    \item Запишем тензорные произведения матриц:
    \begin{equation}
        1\otimes1=
        \left(\begin{array}{cccc}
        1 & 0\\
        0 & 1\\
        \end{array}\right)\otimes
        \left(\begin{array}{cccc}
        1 & 0\\
        0 & 1\\
        \end{array}\right)=\left(\begin{array}{cccc}
        1 & 0 & 0 & 0\\
        0 & 1 & 0 & 0\\
        0 & 0 & 1 & 0\\
        0 & 0 & 0 & 1\\
        \end{array}\right)
    \end{equation}
    \begin{equation}
        \sigma_x\otimes\sigma_x=
        \left(\begin{array}{cccc}
        0 & 1\\
        1 & 0\\
        \end{array}\right)\otimes
        \left(\begin{array}{cccc}
        0 & 1\\
        1 & 0\\
        \end{array}\right)=\left(\begin{array}{cccc}
        0 & 0 & 0 & 1\\
        0 & 0 & 1 & 0\\
        0 & 1 & 0 & 0\\
        1 & 0 & 0 & 0\\
        \end{array}\right)
    \end{equation}
    \begin{equation}
        \sigma_y\otimes\sigma_y=
        \left(\begin{array}{cccc}
        0 & -i\\
        i & 0\\
        \end{array}\right)\otimes
        \left(\begin{array}{cccc}
        0 & -i\\
        i & 0\\
        \end{array}\right)=\left(\begin{array}{cccc}
        0 & 0 & 0 & -1\\
        0 & 0 & 1 & 0\\
        0 & 1 & 0 & 0\\
        -1 & 0 & 0 & 0\\
        \end{array}\right)
    \end{equation}
    \begin{equation}
        \sigma_z\otimes\sigma_z=
        \left(\begin{array}{cccc}
        1 & 0\\
        0 & -1\\
        \end{array}\right)\otimes
        \left(\begin{array}{cccc}
        1 & 0\\
        0 & -1\\
        \end{array}\right)=\left(\begin{array}{cccc}
        1 & 0 & 0 & 0\\
        0 & -1 & 0 & 0\\
        0 & 0 & -1 & 0\\
        0 & 0 & 0 & 1\\
        \end{array}\right)
    \end{equation}
    \begin{equation}
        P_{12}=\left(\begin{array}{cccc}
        1 & 0 & 0 & 0\\
        0 & 0 & 1 & 0\\
        0 & 1 & 0 & 0\\
        0 & 0 & 0 & 1\\
        \end{array}\right)
    \end{equation}
    \begin{multline}
        \left(\begin{array}{cccc}
        1 & 0 & 0 & 0\\
        0 & 1 & 0 & 0\\
        0 & 0 & 1 & 0\\
        0 & 0 & 0 & 1\\
        \end{array}\right)+
        \left(\begin{array}{cccc}
        0 & 0 & 0 & 1\\
        0 & 0 & 1 & 0\\
        0 & 1 & 0 & 0\\
        1 & 0 & 0 & 0\\
        \end{array}\right)+
        \left(\begin{array}{cccc}
        0 & 0 & 0 & -1\\
        0 & 0 & 1 & 0\\
        0 & 1 & 0 & 0\\
        -1 & 0 & 0 & 0\\
        \end{array}\right)+\\
        +\left(\begin{array}{cccc}
        1 & 0 & 0 & 0\\
        0 & -1 & 0 & 0\\
        0 & 0 & -1 & 0\\
        0 & 0 & 0 & 1\\
        \end{array}\right)=
        \left(\begin{array}{cccc}
        2 & 0 & 0 & 0\\
        0 & 0 & 2 & 0\\
        0 & 2 & 0 & 0\\
        0 & 0 & 0 & 2\\
        \end{array}\right)
    \end{multline}
    \begin{equation}\label{eq6}
        \boxed{1\otimes1+\sigma_x\otimes\sigma_x+\sigma_y\otimes\sigma_y+\sigma_z\otimes\sigma_z=2P_{12}}
    \end{equation}
\end{enumerate}
\subsection{Координатный анзац Бете}
\begin{enumerate}
    \item Для начала докажем тождество. Запишем $\vec{\sigma}^{(1)}$ и $\vec{\sigma}^{(2)}$:
    \begin{equation}
        \vec{\sigma}^{(1)}=\vec{\sigma}\otimes1,\quad \vec{\sigma}^{(2)}=1\otimes\vec{\sigma},\quad \vec{\sigma}=(\sigma_x,\sigma_y,\sigma_z)
    \end{equation}
    Скалярное произведение $\vec{\sigma}^{(1)}\vec{\sigma}^{(2)}$:
    \begin{equation}
        \vec{\sigma}^{(1)}\vec{\sigma}^{(2)}=\sigma_x\otimes\sigma_x+\sigma_y\otimes\sigma_y+\sigma_z\otimes\sigma_z
    \end{equation}
    Воспользуемся доказанным тождеством (\ref{eq6}):
    \begin{equation}
        \vec{\sigma}^{(1)}\vec{\sigma}^{(2)}=2P_{12}-1\otimes1=2P_{12}-1
    \end{equation}
    \begin{equation}
        (1-\vec{\sigma}^{(1)}\vec{\sigma}^{(2)})^2=(2\cdot1-2P_{12})^2=4P^2_{12}-8P_{12}+4\cdot1
    \end{equation}
        \begin{equation}
        P^2_{12}=\left(\begin{array}{cccc}
        1 & 0 & 0 & 0\\
        0 & 1 & 0 & 0\\
        0 & 0 & 1 & 0\\
        0 & 0 & 0 & 1\\
        \end{array}\right)
    \end{equation}
    \begin{equation}
        (1-\vec{\sigma}^{(1)}\vec{\sigma}^{(2)})^2=8\cdot1-8P_{12}=4\cdot1-4\vec{\sigma}^{(1)}\vec{\sigma}^{(2)}
    \end{equation}
    \begin{equation}\label{eq7}
        \boxed{\frac{1}{4}(1-\vec{\sigma}^{(1)}\vec{\sigma}^{(2)})^2=1-\vec{\sigma}^{(1)}\vec{\sigma}^{(2)}}
    \end{equation}
    Гамильтониан XXX-цепочки:
    \begin{equation}
        H^{XXX}=-\frac{1}{2}\sum\limits_{k=1}^N(\sigma_x^{(k)}\sigma_x^{(k+1)}+\sigma_y^{(k)}\sigma_y^{(k+1)}+\sigma_z^{(k)}\sigma_z^{(k+1)}-\sigma_0^{(k)}\sigma_0^{(k+1)})
    \end{equation}
    \begin{equation}
        H^{XXX}=-\frac{1}{2}\sum\limits_{k=1}^N(\vec{\sigma}^{(1)}\vec{\sigma}^{(2)}-1)
    \end{equation}
    Воспользуемся доказанным тождеством (\ref{eq7}):
    \begin{equation}
        H^{XXX}=\frac{1}{8}\sum\limits_{k=1}^N(1-\vec{\sigma}^{(1)}\vec{\sigma}^{(2)})^2
    \end{equation}
    Следовательно, все собственные значения гамильтониана XXX-цепочки неотрицательны.
    \item Гамильтониан XXX-цепочки:
    \begin{equation}
        H^{XXX}=-\frac{1}{2}\sum\limits_{k=1}^N(\sigma_x^{(k)}\sigma_x^{(k+1)}+\sigma_y^{(k)}\sigma_y^{(k+1)}+\sigma_z^{(k)}\sigma_z^{(k+1)}-1)=-\sum\limits_{k=1}^NP_{k,k+1}+N
    \end{equation}
    где $P_{k,k+1}$ -- оператор, переставляющий $k$-й и $k+1$-й множители в тензорном произведении.
    \begin{multline}
        [H^{XXX},S_z]=[N-\sum\limits_{k=1}^NP_{k,k+1},\sum\limits_{n=1}^N\sigma_z^{(n)}]=N\sum\limits_{n=1}^N\sigma_z^{(n)}-\sum\limits_{k=1}^NP_{k,k+1}\sum\limits_{n=1}^N\sigma_z^{(n)}-\\
        -\sum\limits_{n=1}^N\sigma_z^{(n)}N+\sum\limits_{n=1}^N\sigma_z^{(n)}\sum\limits_{k=1}^NP_{k,k+1}=\sum\limits_{n=1}^N\sigma_z^{(n)}\sum\limits_{k=1}^NP_{k,k+1}-\sum\limits_{k=1}^NP_{k,k+1}\sum\limits_{n=1}^N\sigma_z^{(n)}=\\=\sum\limits_{k=1}^N(\sigma_z^{(k)}P_{k,k+1}-P_{k,k+1}\sigma_z^{(k)}+\sigma_z^{(k+1)}P_{k,k+1}-P_{k,k+1}\sigma_z^{(k+1)})
    \end{multline}
    Домножим слева на $P_{t,t+1}$:
    \begin{multline}
        \sum\limits_{k=1}^NP_{t,t+1}(\sigma_z^{(k)}P_{k,k+1}-P_{k,k+1}\sigma_z^{(k)}+\sigma_z^{(k+1)}P_{k,k+1}-P_{k,k+1}\sigma_z^{(k+1)})=\\
        =\sum\limits_{k=1}^N((\sigma_z^{(k)}+\sigma_z^{(k+1)})-(\sigma_z^{(k)}+\sigma_z^{(k+1)}))\delta_{tk}=\sum\limits_{k=1}^N(\sigma_z^{(k)}+\sigma_z^{(k+1)}-\sigma_z^{(k)}-\sigma_z^{(k+1)})=0
    \end{multline}
    \begin{equation}
        \boxed{[H^{XXX},S_z]=0}
    \end{equation}
    \item \begin{itemize}
        \item[(a)] Решения уравнения
        \begin{equation}
            H^{XXX}\ket{\Psi}=E\ket{\Psi}
        \end{equation}
        будем искать в виде
        \begin{equation}
            \ket{\Psi}=\ket{\Psi^{(2)}}=\sum\limits_{1\leq k_1\leq k_2\leq N}a(k_1,k_2)\sigma_-^{(k_1)}\sigma_-^{(k_2)}\ket{\Omega}
        \end{equation}
        \begin{equation}
            \left(N-\sum\limits_{k=1}^NP_{k,k+1}\right)\ket{\Psi}=E\ket{\Psi}
        \end{equation}
        \begin{equation}
            \sum\limits_{k=1}^NP_{k,k+1}\ket{\Psi}=(N-E)\ket{\Psi}
        \end{equation}
        Домножим уравнение слева на $\bra{\Omega}\sigma_+^{(n_2)}\sigma_+^{(n_1)}$:
        \begin{equation}
            \bra{\Omega}\sigma_+^{(n_2)}\sigma_+^{(n_1)}\sum\limits_{k=1}^NP_{k,k+1}\ket{\Psi}=\bra{\Omega}\sigma_+^{(n_2)}\sigma_+^{(n_1)}(N-E)\ket{\Psi}
        \end{equation}
        Операторы перестановки пронесём налево, используя правило (его справедливость можно доказать, домножив обе части слева на $P_{k,k+1}$)
        \begin{equation}
            \sigma_+^{(n)}P_{k,k+1}=P_{k,k+1}[\sigma_+^{(n)}+\delta_{kn}(\sigma_+^{(n+1)}-\sigma_+^{(n)})+\delta_{k+1,n}(\sigma_+^{(n-1)}-\sigma_+^{(n)})]
        \end{equation}
        Применив его два раза, получим:
        \begin{multline}
            \bra{\Omega}\sigma_+^{(n_2)}\sigma_+^{(n_1)}P_{k,k+1}= \bra{\Omega}\sigma_+^{(n_2)}\sigma_+^{(n_1)}+\delta_{kn_1}\bra{\Omega}\sigma_+^{(n_2)}(\sigma_+^{(n_1+1)}-\sigma_+^{(n_1)})+\\
            +\delta_{k+1,n_1}\bra{\Omega}\sigma_+^{(n_2)}(\sigma_+^{(n_1-1)}-\sigma_+^{(n_1)})+\delta_{kn_2}\bra{\Omega}(\sigma_+^{(n_2+1)}-\sigma_+^{(n_2)})\sigma_+^{(n_1)}+\\
            +\delta_{k+1,n_2}\bra{\Omega}(\sigma_+^{(n_2-1)}-\sigma_+^{(n_2)})\sigma_+^{(n_1)}+\delta_{kn_1}\delta_{k+1,n_2}\bra{\Omega}(\sigma_+^{(n_2-1)}-\sigma_+^{(n_2)})(\sigma_+^{(n_1+1)}-\sigma_+^{(n_1)})+\\
            +\delta_{kn_2}\delta_{k+1,n_1}\bra{\Omega}(\sigma_+^{(n_2+1)}-\sigma_+^{(n_2)})(\sigma_+^{(n_1-1)}-\sigma_+^{(n_1)})
        \end{multline}
        Просуммируем по $k$ и воспользуемся тем, что
        \begin{equation}
            a(n_1,n_2)=\bra{\Omega}\sigma_+^{(n_2)}\sigma_+^{(n_1)}\ket{\Psi}
        \end{equation}
        Рассмотрим случаи:
        \begin{itemize}
            \item[i.] $n_1<n_2-1$, $(n_1,n_2)\neq(1,N))$.\\
            В этом случае $\delta_{kn_1}\delta_{k+1,n_2}=\delta_{kn_2}\delta_{k+1,n_1}=0$. Получаем уравнение:
            \begin{equation}\label{eq8}
                \boxed{a(n_1+1,n_2)+a(n_1-1,n_2)+a(n_1,n_2+1)+a(n_1,n_2-1)-4a(n_1,n_2)=-Ea(n_1,n_2)}
            \end{equation}
            \item[ii.] $n_1=n_2-1=n$.\\
            В этом случае $\delta_{kn_2}\delta_{k+1,n_1}=0$. Получаем уравнение:
            \begin{equation}\label{eq9}
                \boxed{a(n,n+2)+a(n-1,n+1)-2a(n,n+1)=-Ea(n,n+1)}
            \end{equation}
            \item[iii.] $n_1<n_2-1$, $(n_1,n_2)\neq(1,N))$.\\
            В этом случае $\delta_{kn_1}\delta_{k+1,n_2}=0$. Получаем уравнение:
            \begin{equation}\label{eq10}
                \boxed{a(1,N-1)+a(2,N)-2a(1,N)=-Ea(1,N)}
            \end{equation}
        \end{itemize}
        \item[(b)] Возьмём $a(n_1,n_2)$ в виде:
        \begin{equation}\label{eq11}
            a(n_1,n_2)=Ae^{ip_1n_1+ip_2n_2}+Be^{ip_2n_1+ip_1n_2}
        \end{equation}
        Подставим решение (\ref{eq11}) в (\ref{eq8}):
        \begin{multline}
            Ae^{ip_1(n_1+1)+ip_2n_2}+Be^{ip_2(n_1+1)+ip_1n_2}+Ae^{ip_1(n_1-1)+ip_2n_2}+Be^{ip_2(n_1-1)+ip_1n_2}+\\
            +Ae^{ip_1n_1+ip_2(n_2+1)}+Be^{ip_2n_1+ip_1(n_2+1)}+Ae^{ip_1n_1+ip_2(n_2-1)}+Be^{ip_2n_1+ip_1(n_2-1)}-\\
            -4(Ae^{ip_1n_1+ip_2n_2}+Be^{ip_2n_1+ip_1n_2})=-E(Ae^{ip_1n_1+ip_2n_2}+Be^{ip_2n_1+ip_1n_2})
        \end{multline}
        \begin{equation}
            E=2(2-\cos p_1-\cos p_2)
        \end{equation}
        Вычтем из ($\ref{eq9}$) ($\ref{eq8}$) при $n_1=n_2-1=n<N$, чтобы исключить $E$ (если вычтем из ($\ref{eq10}$) ($\ref{eq8}$), то получим такое же уравнение при $n=N$ в силу периодичности): 
        \begin{equation}
            a(n,n)+a(n+1,n+1)=2a(n,n+1),\quad 1\leq n\leq N
        \end{equation}
        Подставим решение (\ref{eq11}):
        \begin{multline}
            Ae^{ip_1n+ip_2n}+Be^{ip_2n+ip_1n}+Ae^{ip_1(n+1)+ip_2(n+1)}+Be^{ip_2(n+1)+ip_1(n+1)}=\\
            =2(Ae^{ip_1n+ip_2(n+1)}+Be^{ip_2n_1+ip_1(n+1)})
        \end{multline}
        \begin{equation}
            \boxed{\frac{A}{B}=-\frac{1+e^{i(p_1+p_2)}-2e^{ip_1}}{1+e^{i(p_1+p_2)}-2e^{ip_2}}\equiv e^{i\theta(p_1,p_2)}}
        \end{equation}
        Значит, можно записать
        \begin{equation}\label{eq12}
            a(n_1,n_2)=e^{i(p_1n_1+ip_2n_2+\frac{1}{2}\theta(p_1,p_2))}-e^{i(p_1n_1+ip_2n_2-\frac{1}{2}\theta(p_1,p_2))}
        \end{equation}
        \item[(c)] Найдём периодические условия. Первое условие (система $N$-периодична):
        \begin{equation}
            \boxed{a(n_1+N,n_2+N)=a(n_1,n_2)}
        \end{equation}
        Рассмотрим $a(1,n)$ при $n>1$. Амплитуды $a(1,n)$ и $a(1,N-n+2)$ соответствуют состояниям, в которых перевёрнутые спины разделены $n-1$ рёбрами решётки. Они отличаются сдвигом на $n-1$ узел. Для $a$ выполнено условие (см. (\ref{eq12})):
        \begin{equation}
            a(n_1+1,n_2+1)=e^{i(p_1+p_2)}a(n_1,n_2)
        \end{equation}
        Поэтому
        \begin{equation}
            a(n_1+k,n_2+k)=e^{ik(p_1+p_2)}a(n_1,n_2)
        \end{equation}
        Поскольку пара звеньев $(1,n)$ и $(1,N-n+2)$ отличается сдвигом на $n-1$ звеньев, то $a(1,n)$ и $a(1,N-n+2)$ связаны:
        \begin{equation}
            a(1,n)=e^{i(n-1)(p_1+p_2)}a(1,N-n+2)
        \end{equation}
        По аналогии,
        \begin{equation}
            a(n,N+1)=e^{i(n-1)(p_1+p_2)}a(1,N-n+2)
        \end{equation}
        Значит, $a(1,n)=a(n,N+1)$ и второе периодическое условие:
        \begin{equation}
            \boxed{a(n_1,n_2)=a(n_2,n_1+N)}
        \end{equation}
        Условие $a(n_1+N,n_2)=a(n_1,n_2)$ наложить нельзя, т.к. $a$ определена только при $n_1<n_2$.
    \end{itemize}
    \item Гамильтониан $XYZ$-цепочки при $N=2$:
    \begin{equation}
        H^{XYZ}=J_x\sigma_x^{(1)}\sigma_x^{(2)}+J_y\sigma_y^{(1)}\sigma_y^{(2)}+J_z\sigma_z^{(1)}\sigma_z^{(2)}
    \end{equation}
    \begin{equation}
        H^{XYZ}=\left(\begin{array}{cccc}
        0 & 0 & 0 & J_x\\
        0 & 0 & J_x & 0\\
        0 & J_x & 0 & 0\\
        J_x & 0 & 0 & 0\\
        \end{array}\right)+\left(\begin{array}{cccc}
        0 & 0 & 0 & -J_y\\
        0 & 0 & J_y & 0\\
        0 & J_y & 0 & 0\\
        -J_y & 0 & 0 & 0\\
        \end{array}\right)+\left(\begin{array}{cccc}
        J_z & 0 & 0 & 0\\
        0 & -J_z & 0 & 0\\
        0 & 0 & -J_z & 0\\
        0 & 0 & 0 & J_z\\
        \end{array}\right)
    \end{equation}
    \begin{equation}
        H^{XYZ}=\left(\begin{array}{cccc}
        J_z & 0 & 0 & J_x-J_y\\
        0 & -J_z & J_x+J_y & 0\\
        0 & J_x+J_y & -J_z & 0\\
        J_x-J_y & 0 & 0 & J_z\\
        \end{array}\right)
    \end{equation}
    Собственные значения:
    \begin{equation}
        \boxed{E_1=-J_x-J_y-J_z,\quad E_2=-J_x+J_y+J_z,\quad E_3=J_x-J_y+J_z,\quad E_4=J_x+J_y-J_z}
    \end{equation}
    Собственные векторы:
    \begin{equation}
        \boxed{\Psi_1=\left(\begin{array}{c}
        0\\
        -1\\
        1\\
        0\\
        \end{array}\right),\quad \Psi_2=\left(\begin{array}{c}
        -1\\
        0\\
        0\\
        1\\
        \end{array}\right),\quad \Psi_3=\left(\begin{array}{c}
        1\\
        0\\
        0\\
        1\\
        \end{array}\right),\quad \Psi_4=\left(\begin{array}{c}
        0\\
        1\\
        1\\
        0\\
        \end{array}\right)}
    \end{equation}
\end{enumerate}
\section{Некоторые задачи СТО и ОТО (Б. Еремин)}
\subsection{Некоторые задачи СТО и ОТО I}
\begin{enumerate}
\item Рассмотрим два кольца, одно из которых покоится, а другое движется относительно первого со скоростью $V$ навстречу ему в системе $K'$. Предположим, что движущееся кольцо уменьшает свои поперечные размеры ($L''<L$). Тогда в системе $K'$ второе кольцо пройдёт внутри первого, в системе $K$ (которая движется вместе со вторым кольцом со скоростью $V$) -- наоборот. Нарушается объективная реальность: может произойти либо одно, либо другое. Если движущееся кольцо увеличивает свои поперечные размеры ($L''>L$), то происходит то же самое. Значит остаётся одно -- размеры не изменяются, $L''=L$.
\item Пусть в системе $K$ одновременно произошли вспышки света в точках с координатами $x_1=0$ и $x_2=x$ в момент времени $t_1=t_2=0$. Пусть в системе $K'$ события произойдут в моменты времени: $t'_1=0$, $t'_2=\Delta t'>0$. Пусть $\tau$ -- время встречи сигналов по часам в $K'$. Левый сигнал идёт время $\tau$ и проходит расстояние
\begin{equation}
c\tau=\frac{x'}{2}+V\tau
\end{equation}
Правый сигнал проходит расстояние
\begin{equation}
c(\tau-\Delta t')=\frac{x'}{2}-V(\tau-\Delta t')
\end{equation}
Итого получаем
\begin{equation}
\Delta t'=\frac{Vx'}{c^2-V^2}=\frac{Vx}{c^2\sqrt{1-\frac{V^2}{c^2}}}
\end{equation}
Где последнее равенство записано с учётом лоренцева сокращения длин (см. пункт 4).
\item Пусть у нас есть часы, сделанные из 2 параллельных зеркал, и они движутся параллельно плоскости этих зеркал в системе $K'$ со скоростью $V$. Пусть расстояние между зеркалами -- $L$. В ней световой пучок пройдёт между испусканием и прибытием светового пучка к одному зеркалу путь
\begin{equation}
c\Delta t' = 2\sqrt{\left(\frac{V\Delta t'}{2}\right)^2+L^2}
\end{equation}
Из этого можно выразить время $\Delta t'$:
\begin{equation}
\Delta t'=\frac{2L}{c\sqrt{1-\frac{V^2}{c^2}}}
\end{equation}
Время прохождения сигнала в системе $K$:
\begin{equation}
\Delta t=\frac{2L}{c}
\end{equation}
Итого получаем:
\begin{equation}
\Delta t' = \frac{\Delta t}{\sqrt{1-\frac{V^2}{c^2}}}
\end{equation}
\item Используем результаты эксперимента Майкельсона-Морли. Пусть длина пути в вертикальном направлении -- $L$ (она не меняется при движении, см. пункт а), в горизонтальном -- $L'$. Пусть свет идёт в горизонтальном направлении: туда -- время $t_1$, обратно -- $t_2$, общее -- $t$; в вертикальном направлении: туда -- время $t'_1$, обратно -- $t'_2$, общее -- $t'$. Тогда мы можем записать следующие уравнения:
\begin{equation}
ct_1=L'+Vt_1 \quad ct_2=L'-Vt_2
\end{equation}
\begin{equation}
t = t_1 + t_2 = \frac{2L'c}{c^2-V^2}
\end{equation}
\begin{equation}
(ct'_1)^2 = L^2+(Vt'_1)^2\quad (ct'_2)^2 = L^2+(Vt'_2)^2
\end{equation}
\begin{equation}
t' = t'_1 + t'_2 = \frac{2L}{\sqrt{c^2-V^2}}
\end{equation}
Из эксперимента Майкельсона-Морли следует $t=t'$:
\begin{equation}
L' = L\sqrt{1-\frac{V^2}{c^2}}
\end{equation}
\item Суммируя пункты (1) - (4), мы можем получить:
\begin{equation}
\begin{cases}
x' = Vt'+x\sqrt{1-\frac{V^2}{c^2}}\\
t' = \frac{t}{\sqrt{1-\frac{V^2}{c^2}}}+\Delta t'(x) = \frac{t+\frac{Vx}{c^2}}{\sqrt{1-\frac{V^2}{c^2}}}
\end{cases}
\end{equation}
Запишем преобразования Лоренца:
\begin{equation}
\begin{cases}
x' = \frac{x+Vt}{\sqrt{1-\frac{V^2}{c^2}}}\\
y' = y\\
z' = z\\
t' = \frac{t+\frac{Vx}{c^2}}{\sqrt{1-\frac{V^2}{c^2}}}
\end{cases}
\end{equation}
Перепишем в более симметричном виде:
\begin{equation}
\begin{cases}
x' = \Gamma(x+c\beta t)\\
y' = y\\
z' = z\\
ct' = \Gamma(ct+\beta x)
\end{cases}
\end{equation}
где $\Gamma=\frac{1}{\sqrt{1-\frac{V^2}{c^2}}}$ -- Лоренц-фактор,
$\beta=\frac{V}{c}$ -- безразмерная скорость.
\end{enumerate}
\subsection{Некоторые задачи СТО и ОТО II}
\begin{enumerate}
    \item Предположим, что $e$, $b_{i_1}$, $b_{i_1}b_{i_2}$, ..., $b_1...b_n$, где $i_j\in \overline{1,n}$ и $i_j<i_{j+1}$ -- базис алгебры $\text{CL}(n,\mathbb{C})$.\\
    Покажем, что любой элемент алгебры может быть выражен через них. Поскольку $b_i$ -- генераторы, то $\forall U\in\text{CL}(n,\mathbb{C})$ выполняется
    \begin{equation}
        U=\sum\limits_{i_j}u_{i_1,...i_n}\prod_{j=1}^nb_{i_j}^{\alpha_{i_j}}
    \end{equation}
    Т.к. $\{b_i,b_j\}=2g_{ij}e$, то $b_i^2=g_{ii}e$.\\
    Если $\alpha_{i_j}=2\beta_{i_j}$ -- чётное, то $b_{i_j}^{\alpha_{i_j}}=b_{i_j}^{2\beta_{i_j}}=g_{i_ji_j}^{\beta_{i_j}}e$. Если $\alpha_{i_j}=2\beta_{i_j}+1$ -- нечётное, то $b_{i_j}^{\alpha_{i_j}}=b_{i_j}^{2\beta_{i_j}}b_{i_j}=g_{i_ji_j}^{\beta_{i_j}}b_{i_j}$. Сумма разобьётся на слагаемые, содержащие $e$ и $b_{i_j}$:
    \begin{equation}
        U=\sum\limits_{i_j}u'_{i_1,...i_n}\prod_{j=1}^nb_{i_j}^{\gamma_{i_j}},\quad \gamma_{i_j}\in\{0,1\}
    \end{equation}
    Если в произведении случился беспорядок между соседними множителями ($i_j>i_{j+1}$), то переставим их:
    \begin{equation}
        b_{i_j}b_{i_{j+1}}=2g_{i_ji_{j+1}}e-b_{i_{j+1}}b_{i_j}
    \end{equation}
    Таким образом, любой элемент алгебры может быть выражен через $e$, $b_{i_1}$, $b_{i_1}b_{i_2}$, ..., $b_1...b_n$. Т.к. любая нетривиальная линейная комбинация этих элементов не равна 0, то это выражение единственно и это базис.\\
    Найдём количество элементов в базисе (размерность $\text{CL}(n,\mathbb{C})$). Число вариантов выбора $\{i_j\}$ с $j\leq k$, где $0\leq k\leq n$ равно числу сочетаний $C_n^k$. Бином Ньютона:
    \begin{equation}
        (1+1)^n=\sum_{k=0}^nC_n^k1^k=\sum_{k=0}^nC_n^k
    \end{equation}
    \begin{equation}
        \boxed{\text{dim\;CL}(n,\mathbb{C})=2^n}
    \end{equation}
    \item Будем искать $S(\Lambda)$ в виде:
    \begin{equation}
        S(\Lambda)=E_{4\times 4}+\omega_{\mu\nu}\Gamma^{\mu\nu}
    \end{equation}
    Запишем преобразование, которому соответствует $S(\Lambda)$:
    \begin{equation}
        S(\Lambda)\Lambda_\nu^\mu\gamma^\nu=\gamma^\mu S(\Lambda)
    \end{equation}
    Подставим $S(\Lambda)$:
    \begin{equation}
        (E_{4\times 4}+\omega_{\alpha\beta}\Gamma^{\alpha\beta})(\delta_\nu^\mu+\omega_\nu^\mu)\gamma^\nu=\gamma^\mu(E_{4\times 4}+\omega_{\alpha\beta}\Gamma^{\alpha\beta})
    \end{equation}
    \begin{equation}
        (E_{4\times 4}+\omega_{\alpha\beta}\Gamma^{\alpha\beta})(\gamma^\mu+\omega_\nu^\mu\gamma^\nu)=\gamma^\mu(E_{4\times 4}+\omega_{\alpha\beta}\Gamma^{\alpha\beta})
    \end{equation}
    \begin{equation}
        \omega_\nu^\mu\gamma^\nu+\omega_{\alpha\beta}\Gamma^{\alpha\beta}\gamma^\mu+\omega_{\alpha\beta}\Gamma^{\alpha\beta}\omega_\nu^\mu\gamma^\nu=\gamma^\mu\omega_{\alpha\beta}\Gamma^{\alpha\beta}
    \end{equation}
    Т.к. $|\omega_\nu^\mu|\ll 1$, то можно пренебречь слагаемым со вторым порядком малости:
    \begin{equation}
        \omega_\nu^\mu\gamma^\nu=\gamma^\mu\omega_{\alpha\beta}\Gamma^{\alpha\beta}-\omega_{\alpha\beta}\Gamma^{\alpha\beta}\gamma^\mu
    \end{equation}
    Будем искать $\Gamma^{\mu\nu}$ пропорциональной коммутатору $\Gamma^{\mu\nu}=C[\gamma^\mu,\gamma^\nu]$:
    \begin{equation}
        \omega_\nu^\mu\gamma^\nu=C\gamma^\mu\omega_{\alpha\beta}(\gamma^\alpha\gamma^\beta-\gamma^\beta\gamma^\alpha)-C\omega_{\alpha\beta}(\gamma^\alpha\gamma^\beta-\gamma^\beta\gamma^\alpha)\gamma^\mu
    \end{equation}
    Учтём, что $\gamma^\alpha\gamma^\beta-\gamma^\beta\gamma^\alpha\neq0$ только при $\alpha\neq\beta$ и $\gamma^\alpha\gamma^\beta+\gamma^\beta\gamma^\alpha=0$.
    \begin{equation}
        \omega_\nu^\mu\gamma^\nu=2C\omega_{\alpha\beta}(\gamma^\mu\gamma^\alpha\gamma^\beta-\gamma^\alpha\gamma^\beta\gamma^\mu)
    \end{equation}
    При $\mu\neq\alpha$, $\mu\neq\beta$ $\gamma^\mu\gamma^\alpha\gamma^\beta-\gamma^\alpha\gamma^\beta\gamma^\mu=\gamma^\mu\gamma^\alpha\gamma^\beta-\gamma^\mu\gamma^\alpha\gamma^\beta=0$. Значит
    \begin{equation}
        \omega_\nu^\mu\gamma^\nu=2C\omega_{\mu\beta}(\gamma^\mu\gamma^\mu\gamma^\beta-\gamma^\mu\gamma^\beta\gamma^\mu)-2C\omega_{\alpha\mu}(\gamma^\mu\gamma^\alpha\gamma^\mu-\gamma^\alpha\gamma^\mu\gamma^\mu)
    \end{equation}
    \begin{equation}
        \omega_\nu^\mu\gamma^\nu=4C(\omega_{\mu\beta}\eta^{\mu\mu}\gamma^\beta-\omega_{\alpha\mu}\eta^{\mu\mu}\gamma^\alpha)=8C\omega_{\mu\nu}\eta^{\mu\mu}\gamma^\nu=8C\omega^\mu_\nu\gamma^\nu
    \end{equation}
    Значит, $C=\frac{1}{8}$.
    \begin{equation}
        \boxed{\Gamma^{\mu\nu}=\frac{1}{8}[\gamma^\mu,\gamma^\nu]}\rightarrow\boxed{S(\Lambda)=E_{4\times 4}+\frac{1}{8}\omega_{\mu\nu}[\gamma^\mu,\gamma^\nu]}
    \end{equation}
\end{enumerate}
\section{Эффект Унру (П. Анемпотистов)}
Функция отклика:
\begin{equation}
    \Pi(\omega)=\int\limits_{-\infty}^\infty dte^{-i\omega t}\braket{\phi(t,\vec{x}(t))\phi(0,0)}
\end{equation}
где $\braket{\phi(t,\vec{x}(t))\phi(t,\vec{x}'(t))}=\frac{1}{4\pi^2}\frac{1}{-(t-t'-i\epsilon)^2+(\vec{x}-\vec{x}')^2}$ -- корреляционная функция, где $\epsilon$ мало.
\begin{equation}
    \Pi(\omega)=\int\limits_{-\infty}^\infty dt\frac{e^{-i\omega t}}{4\pi^2(x^2-(t-i\epsilon)^2)}
\end{equation}
\begin{enumerate}
    \item Перейдём в систему покоя детектора ($x=0$):
    \begin{equation}
        \Pi_\text{inertial}(\omega)=-\frac{1}{4\pi^2}\int\limits_{-\infty}^\infty dt\frac{e^{-i\omega t}}{(t-i\epsilon)^2}
    \end{equation}
    Подынтегральная функция имеет особую точку 2 порядка $t=i\epsilon$.
    \begin{equation}
        \text{Res}\left(\frac{e^{-i\omega t}}{(t-i\epsilon)^2}\right)(i\epsilon)=\lim\limits_{t\rightarrow i\epsilon}\left(\frac{de^{-i\omega t}}{dt}\right)=-i\omega e^{\epsilon\omega}
    \end{equation}
    \begin{equation}
        \Pi_\text{inertial}(\omega)=-\frac{2\pi i}{4\pi^2}(-i\omega \theta(-\omega))
    \end{equation}
    где $\theta$ -- тета-функция Хевисайда.
    \begin{equation}
        \boxed{\Pi_\text{inertial}(\omega)=-\frac{\omega}{2\pi}\theta(-\omega)}
    \end{equation}
    \item При равноускоренном движении детектора:
    \begin{equation}
        x=\frac{1}{\alpha}\cosh\tau,\quad t=\frac{1}{\alpha}\sinh\tau,\quad\tau(-\infty,\infty)
    \end{equation}
    %\begin{equation}
     %   \Pi_\text{accelerated}(\omega)=\frac{1}{4\pi^2}\int\limits_{-\infty}^\infty dt\frac{e^{-\frac{i\omega\sinh\tau}{\alpha}}}{\frac{1}{\alpha^2}\cosh^2\tau-(\frac{1}{\alpha}\sinh\tau-i\epsilon)^2}=\frac{1}{4\pi^2}\int\limits_{-\infty}^\infty d\tau\frac{\cosh\tau e^{-i\omega t}}{\frac{1}{\alpha^2}\cosh^2\tau-(\frac{1}{\alpha}\sinh\tau-i\epsilon)^2}
    %\end{equation}
    \begin{equation}
        \Pi_\text{accelerated}(\omega)=-\frac{\alpha}{16\pi^2}\int\limits_{-\infty}^\infty d\tau\frac{e^{-\frac{i\omega\tau}{\alpha}}}{\sinh^2(\frac{\tau}{2}-i\epsilon)}
    \end{equation}
    Подынтегральная функция имеет счётное число особых точек 2 порядка $t=2i\epsilon+2i\pi k$, где $k\in\mathbb{Z}$.
    \begin{multline}
        \text{Res}\left(\frac{e^{-\frac{i\omega\tau}{\alpha}}}{\sinh^2(\frac{\tau}{2}-i\epsilon)}\right)(2i\epsilon+2i\pi k)=\text{Res}\left(\frac{4e^{-\frac{i\omega\tau}{\alpha}}}{(\tau-2i\epsilon)^2}\right)(2i\epsilon+2i\pi k)=\\
        =\lim\limits_{\tau\rightarrow 2i\epsilon+2i\pi k}\left(4\frac{de^{-\frac{i\omega\tau}{\alpha}}}{dt}\right)=-\frac{4i\omega}{\alpha}e^{\frac{2\omega}{\alpha}(\epsilon+\pi k)}
    \end{multline}
    \begin{equation}
        \Pi_\text{accelerated}(\omega)=-\frac{8\pi\omega\alpha}{16\pi^2\alpha}\sum\limits_{k=0}^\infty e^{\frac{2\omega}{\alpha}(\epsilon-\pi k)}=\frac{\omega}{2\pi}\frac{1}{e^{\frac{2\pi\omega}{\alpha}}-1}
    \end{equation}
    где в последнем равенстве посчитана сумма геометрической прогрессии.
    \begin{equation}
        \boxed{\Pi_\text{accelerated}(\omega)=\frac{\omega}{2\pi(e^{\frac{2\pi\omega}{\alpha}}-1)}}
    \end{equation}
\end{enumerate}
\end{document}
